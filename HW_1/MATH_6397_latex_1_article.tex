%%%%Select the class of document
\documentclass{amsart} %type/class of article
%\documentclass{article}
%\documentclass{book}
%\documentclass{letter}

%%%%Select packages
\usepackage{amsthm,amsmath,amssymb} %math packages (always include)
\usepackage{geometry} %can be used to modify page dimensions, etc.
\usepackage{graphicx} %figure
\usepackage{float} %figure position in pdf
\usepackage{multirow} %table with multirow
\usepackage[labelfont=rm]{subcaption} %subcaption of figure
\usepackage[numbers]{natbib} %management of bibliography
\usepackage{listings} %use to list command or else in block
\usepackage{xcolor}  %allow to display color
%%%%% End Select packages

%%%%Customize commands
\usepackage{mydef}  %add new command from a file called mydef.sty
%Note that you can copy the content of mydef.sty here and comment the above line.
%%%%End Customize commands

%%%% Define a style used to display a block of Latex code
\lstdefinestyle{TexStyle}{
language={[LaTeX]TeX},
frame=single,
backgroundcolor=\color{white},
basicstyle=\small\ttfamily,
morekeywords={maketitle,includegraphics},
keywordstyle=\color{blue},  
commentstyle=\color{gray},
stringstyle=\color{black}
}


%%%% Set up title page information
\title{CAAM 519: Computational Science I \\
Latex Example 1.}
\author{Lo\"{i}c Cappanera}
%\date{\today}
%%%% End Set up title page information


%%% Begin document
\begin{document}

%%%% Write an abstract if needed
\begin{abstract}
Recently, I have been reading about optimal way to train neural ODEs from \citep{finlay2020}.
\end{abstract}

%%%% Make title page
\maketitle

\section{Tree of the article} \label{sec:latex_structure}

An article starts with a section, then subsection (then subsubsection if required). A book start with chapters.

\subsection{Subsection}
This is my first subsection.

\subsection{Subsection}
This is my second subsection.
\subsubsection{Subsubsection}
\paragraph{Hello this is a paragraph in a subsubsection.}

\paragraph{Hello this is another paragraph}
\subsubsection{Subsubsection}
You do not need to use the command paragraph to write text.

\noindent Here is a new paragraph without using the paragrah command.

\section{Citing scientific article or book} \label{sec:latex_cite_biblio}

%You can add \noindent if you dont want indenting at the begining of your paragraph.
The Latex companion of Mittelbach et al\cite{mittelbach2004latex}

The C programming language of Kernighan et al\cite{kernighan2006c}

The C++ programming language of Stroustrup
\cite{stroustrup2013c++}

\section{Include mathematics formulae} \label{sec:latex_math}

\subsection{Equations} You can write mathematical formulae using  $\$ formula \$ $ as follows:
$a=b$.

It is possible to center the formulae.
%center the math formulae as follows:
$$a=b$$
%or
\[  a = b\]

Write an equation.
\begin{equation}
a=b
\end{equation}

Write an equation that is longer than one line.
\begin{multline}
a + b + c + d + e + f +g + h + i + j + k + l + m + n
\\
- o - p -q -u - r -s - t - v -w -x -y =z.
\end{multline}

Align  multiple equations (see in comment how to remove the numbering of the equations).
\begin{align}
\label{eq:first_align_eq}
a 	&=b,
\\
c	&=d.
\end{align}
%remove numbering of the aligned equations:
%\begin{align*}
%a 	&=b,
%\\
%c	&=d.
%\end{align*}

To be able to label each aligned equations, use subequations as follows:
\begin{subequations}\label{eq:Navier_Stokes}
\begin{align}
 &\partial_t \rho   + \DIV \textbf{m} =0 \label{eq:mass_equation}\\ 
 &\partial_t (\rho \textbf{u})   + \DIV( \textbf{m}{\otimes}\textbf{u})  
-  \DIV (\eta\GRAD (\textbf{u})) + \GRAD p 
 =   \textbf{f}, 
\label{eq:momentum_equation}\\
&\DIV \textbf{u}   =    0,  \label{eq:incompressibility_equation} 
\end{align}
\end{subequations}

\subsection{Tables} Table can be used to contain tabular or array. The main difference is that an array is written in a mathematical mode (between $\$$ sign).

Here is an array.
\begin{table}[h]
\[
\begin{array}{|c|c|c|}
\hline
First row & \cos(e^x) & x_1+ x_2 \\
\hline
\text{Second Row} & a & b \\
\hline
\end{array}
\]
\caption{My first array}
\end{table}

Here is a tabular.
\begin{table}[h]
\begin{tabular}{|c|c|c|}
\hline
First row & $\cos(e^x)$ & $x_1+ x_2$ \\
\hline
Second Row & a & b \\
\hline
Third row &c  & d\\
\hline
\end{tabular}
\caption{My first tabular}
\label{tab:tabular}
\end{table}

Here is a the block of code used to generate the table~\ref{tab:tabular}.
\begin{lstlisting}[style=TexStyle]
\begin{table}[h]
\begin{tabular}{|c|c|c|}
\hline
First row & $\cos(e^x)$ & $x_1+ x_2$ \\
\hline
Second Row & a & b \\
\hline
Third row &c  & d\\
\hline
\end{tabular}
\caption{My first tabular}
\label{tab:tabular}
\end{table}
\end{lstlisting}

Here is a tabular with multirow and multicolumn:
\begin{table}[h]
\centering
\begin{tabular}{|c|c|c|c|c|c|c|} \hline\multicolumn{3}{|c|}{$\textbf{L}^2$-norm of error}
& \multicolumn{2}{|c|}{Velocity}
& \multicolumn{2}{|c|}{Pressure} \\ \hline \hline 
&  $h$  & $n_\text{df}$ & Error  & Rate  & Error  & Rate   \\ \hline
\multirow{5}{*}{Test 1}
&$0.1$    &  270& 2.22E-3& --  & 5.03E-4& --   \\ \cline{2-7} 
&$0.05$   &  986& 7.90E-4& 1.60& 2.14E-4& 1.32\\  \cline{2-7} 
&$0.025$  & 3810& 3.66E-4& 1.14& 1.04E-4& 1.07\\  \cline{2-7} 
&$0.0125$ &14993& 1.82E-4& 1.02& 5.16E-5& 1.02\\ \cline{2-7} 
&$0.00625$&59628& 5.80E-5& 1.65& 1.65E-5& 1.65\\
\hline\hline
\multirow{5}{*}{Test 2}
&$0.1$    &  270& 1.20E-2& --  & 9.12E-2& --  \\ \cline{2-7} 
&$0.05$   &  986& 1.48E-3& 2.99& 3.88E-2& 1.32\\  \cline{2-7} 
&$0.025$  & 3810& 3.96E-4& 1.96& 1.43E-3& 1.48\\  \cline{2-7} 
&$0.0125$ &14993& 1.77E-4& 1.18& 6.72E-3& 1.10\\  \cline{2-7} 
&$0.00625$&59628& 5.63E-5& 1.66& 2.14E-3& 1.66\\ \hline 
\end{tabular} 
\caption{Examples of table with multirow and multicolumn. h is hte mesh size qnd $n_\text{df}$ the number of degree of freedom. Table from Cappanera et al 2017 (IJNMF).}
\label{tab:tabular_multirow}
\end{table}
  

\section{Include figures} \label{sec:latex_figure}
One figure
\begin{figure}[h]
\includegraphics[width=0.4\textwidth]{FIGS/J0_2d4_85.png}
\caption{Snapshot interface between two fluids. Image from Cappanera et al 2017 (IJNMF).}
\label{fig:first_fig}
\end{figure}


Four figures aligned horizontally using minipage and subcaption.
\begin{figure}[h]
\begin{minipage}[b]{0.24\linewidth}
\includegraphics[width=0.9\textwidth]{FIGS/J0_2d4_82.png}
\subcaption{$t{=}8$}
\end{minipage}\hfil
\begin{minipage}[b]{0.24\linewidth}
\includegraphics[width=0.9\textwidth]{FIGS/J0_2d4_83.png}
\subcaption{$t{=}8.2$}
\end{minipage}\hfil
\begin{minipage}[b]{0.24\linewidth}
\includegraphics[width=0.9\textwidth]{FIGS/J0_2d4_84.png}
\subcaption{$t{=}8.4$}
\end{minipage} \hfil
\begin{minipage}[b]{0.24\linewidth}
\includegraphics[width=0.9\textwidth]{FIGS/J0_2d4_85.png}
\subcaption{$t{=}8.6$}
\end{minipage}
\caption{Evolution of interface between two fluids. Images from Cappanera et al 2017 (IJNMF).}
\label{fig:fig_with_minipage}
\end{figure}


%%% Add bibliography if any
\bibliographystyle{abbrvnat}  %select the style
\bibliography{biblio} % select the files (without the .bib)

%%%% End document
\end{document}
